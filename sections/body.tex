\section{Body} \label{sec:body}


The body of your thesis usually consists of a sequence of sections and sunsections that tell the story of your problem, approach and results. 

We're using this section to exemplify some of the conventions you should follow for integrating figures, references etc.

\subsection{Footnotes and References} \label{sec:body_refs}

References to external work are your way provide additional information to the reader.

Footnotes are used for any information that isn't crucial for the text but could be used by a more interested reader to get more information \footnote{It's a good idea to keep them short and to the point}. They may also contain links \footnote{e.g., to \url{http://isitfriday.org}}.

References are your way to refer to existing scientific work. They shouldn't be limited to the state of the art section. They can be used whenever you want to establish a fact that has already been published or if you want to point the reader to a place where they can find additional information.
Throughout your thesis you should include keys to the respective publications right after you use content from them. For example, this would look like this: \cite{greenwade93} \cite{einstein} .


\subsection{Tables and Figures} \label{sec:body_tables}

Tables and Figures are your way of illustrating your work. You can find an example for a table in Table \ref{table:example_table} and an example of a figure in Figure \ref{figure:example_figure}.

\begin{table}[htb]
    \centering
    \caption{Some table in the second section}
    \begin{tabular}{||c c c c||} 
 \hline
 Col1 & Col2 & Col2 & Col3 \\ [0.5ex] 
 \hline\hline
 1 & 6 & 87837 & 787 \\ 
 2 & 7 & 78 & 5415 \\
 3 & 545 & 778 & 7507 \\
 4 & 545 & 18744 & 7560 \\
 5 & 88 & 788 & 6344 \\ [1ex] 
 \hline
\end{tabular}
    \label{table:example_table}
\end{table}

\begin{figure}
    \centering
      \includegraphics[width=300]{example-image-a}
      \caption{Some big figure example with a very long text, to test the spacing in multi-line captions.}
      \label{figure:example_figure}
    \end{figure}

They say a figure says more than a thousand words. While that is true, you can't always control which words the reader will read from a figure. Thus, you should always explicitly reference and discuss any tables or figures in your text to make sure the readers interpretation aligns with your intention and to make any conclusions and arguments you draw from the figure explicit.

Bigger figures like \autoref{figure:example_figure} will go on the next page, if the remaining space is too small. That's why it's good to always reference them.
