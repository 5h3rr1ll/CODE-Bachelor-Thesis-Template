% Specify date variables upfront
\newdate{dateOne}{01}{03}{2021}
\newdate{dateTwo}{01}{04}{2021}
\newdate{dateThree}{09}{04}{2021}
\newdate{dateFour}{16}{04}{2021}
\newdate{dateFive}{21}{04}{2021}
\newdate{dateSix}{28}{04}{2021}
\newdate{dateSeven}{07}{05}{2021}
\newdate{dateEight}{28}{04}{2021}
\newdate{dateNine}{28}{04}{2021}
\newdate{dateTen}{28}{04}{2021}

\section{Introduction} \label{sec:introduction}

The introduction should give the reader a short introduction into the topic you have researched and the structure of the thesis. It is usual to start out the introduction with a short motivation of the general problem, leading over to the specific aspect tackled by your work. You should also give a summary of the research approach you are taking to tackle this aspect. This enables the reader to understand the goals of your research and the approach you are taking. Results are usually not part of this section.

The introduction should end with a short summary of the structure of your thesis. This enables the reader to understand what to expect of the document itself and where to search for specific points. It is usual here to list all of the main sections and quickly describe their content in a section. E.g., if I were to decribe the content of this template I would tell you that Section \ref{sec:body} will tell you about the body of the thesis and introduce and exempify some of the conventions you should be using. Section \ref{sec:conclusion} will conclude the template by telling you how you should approach the conclusion.
